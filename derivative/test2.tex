\documentclass{jsarticle}
\usepackage{amsmath}
\usepackage{bm}
\usepackage{xparse}

\makeatletter

\renewcommand{\d}{\textrm{d}}

% 常微分マクロ
\NewDocumentCommand\odr{m O{} m}{%
	\frac{%
		\d^{#2} #3%
	}{%
		\d #1^{#2}%
	}%
}

% 階数を計算するためのカウンタ
\newcounter{@derivative@totalOrder}
% 階数を計算するかどうか
\newif\if@derivative@calcOrder@
% 階数を計算
\newcommand{\@derivative@addToTotalOrder}[1]{%
	\ifx\hfuzz#1\hfuzz% 空の場合
		\addtocounter{@derivative@totalOrder}{1}%
	\else%
		\addtocounter{@derivative@totalOrder}{#1}%
	\fi%
}
% 階数を表示
\newcommand{\@derivative@printTotalOrder}{%
	\ifnum\value{@derivative@totalOrder}=1% 階数が1の場合
		% 空
	\else%
		\the@derivative@totalOrder% 階数を表示
	\fi%
}
% 内部変数をリセット
\newcommand{\@derivative@resetInnerVar}{%
	\setcounter{@derivative@totalOrder}{0}%
	%
	% 偏微分の分母と分子を一時的に保存
	\def\@derivative@denominator{}% 分母
	\def\@derivative@numerator{}% 分子
	%
	\def\@derivative@arg@@@{}%
}

% version1
% https://zrbabbler.hatenablog.com/entry/20110903/1315024613
% 一つずつ読み取っていく
\NewDocumentCommand\@pdr{m O{} m}{%
	\renewcommand{\@derivative@arg@@@}{#3}%
	\ifx#3\@pdr@end% 分母処理(最終)
		\renewcommand{\@derivative@numerator}{%
			\partial^{%
 				\if@derivative@calcOrder@%
 					\@derivative@printTotalOrder%
 				\else%
 					#2%
 				\fi%
			} #1%
		}%
	\else% 分子処理
		\if@derivative@calcOrder@%
			\@derivative@addToTotalOrder{#2}% 階数に足しこむ
		\fi%
		%
		\expandafter\renewcommand% 分母更新
		\expandafter\@derivative@denominator%
		\expandafter{%
			\@derivative@denominator \partial #1^{#2}%
		}%
		%
		\expandafter\expandafter\expandafter\@pdr% 再帰 \fiを正常に動かすためにexpandafterをつける
		\expandafter\expandafter\expandafter{%
		\expandafter\@derivative@arg@@@%
		\expandafter}%
	\fi%
}

% 終了判定用
\newcommand{\@pdr@end}{\@pdr@end@}

\NewDocumentCommand{\pdr}{s m}{%
	\@derivative@resetInnerVar% 内部変数の初期化
	\IfBooleanTF{#1}{%
		% starが指定された場合階数計算をしない
		\@derivative@calcOrder@false%
	}{%
		\@derivative@calcOrder@true%
	}%
	\@pdr#2\@pdr@end%
	\frac{\@derivative@numerator}{\@derivative@denominator}%
}


\makeatother

%%%%%%%%%%%%%%%%%%%%%%%%%%%%%%%%%%%%%%%%%%%%%%%%%%%%%%%%%%%%%%%%%%%%%%%%%%%%%%%%%%%%%%%%%%%%%%%%%%
% physicsパッケージの偏微分マクロは$\displaystyle \frac{\partial^5 f}{\partial x^2 \partial y^3}$
% のようなものに対応しておらず、
% 使い勝手が微妙に悪いので、自作する
%%%%%%%%%%%%%%%%%%%%%%%%%%%%%%%%%%%%%%%%%%%%%%%%%%%%%%%%%%%%%%%%%%%%%%%%%%%%%%%%%%%%%%%%%%%%%%%%%%
% test.texにて凡そ実装の方向性が固まった(version1を採用)ので、
% そこからブラッシュアップ
%


\begin{document}
\tracingonline=1
\tracingmacros=1
\tracingcommands=1

常微分マクロ\verb|\odr|
\begin{align}
	\odr{x}{f} \\
	\odr{x}[2]{f} \\
	\odr{x}[n]{f} \\
	\odr{x_1}{f} \\
	\odr{x_1}{y_2} \\
	\odr{t}{\bm{u}} \\
	\odr{\bm{x}}{f}
\end{align}

偏微分マクロ\verb|\pdr| version 1
\begin{align}
1		&\	\pdr{{x}{f}} \\ 				% pass
2		&\	\pdr{{x}[2]{f}} \\ 				% pass
3		&\	\pdr{{x}{y}{f}} \\ 				% pass
4		&\	\pdr{{x}[2]{y}{f}} \\ 			% pass
5		&\	\pdr{{x}[2]{y}[3]{f}} \\ 		% pass
6		&\	\pdr{{x_1}{y}} \\ 				% pass
7		&\	\pdr{{x}{f_2}} \\ 				% pass
8		&\	\pdr{{x_1}{t}{f}} \\ 			% pass
9		&\	\pdr{{x_1}{x_2}{f}} \\ 			% pass
10		&\	\pdr{{\bm{x}}{f}} \\ 			% pass
11		&\	\pdr{{\bm{x}}{\bm{y}}{f}} \\ 	% pass
12		&\	\pdr{{\bm{x}_1}{\bm{y}_2}{f}} \\ %
13		&\	\pdr{{x}[2]{\bm{u}}} \\ 		% pass
14		&\	\pdr{{x}{\bm{u}}} + \pdr{{y}{\bm{u}}} \\ %
15		&\	\pdr{{x}{}} \\					% pass
16		&\	\pdr{{x_1}{}} \\				% pass
17		&\	\pdr{{\bm{x}}{}} \\				% pass
\end{align}

\begin{align}
1		&\	\pdr*{{x}[n]{f}[n]} \\			% pass
2		&\	\pdr*{{x}[n]{}[n]} \\			% pass
3		&\	\pdr*{{x}[n]{y}[m]{f}[n+m]} \\	% pass
4		&\	\pdr*{{x}[1]{y}[1]{f}[1+1]} \\	% pass
5		&\ 	\pdr*{{x}{y}[n]{f}[n+1]} \\		% pass
\end{align}
%
%\pdr{
%	{x}{y}{f}
%}
%
%\pdr{{x}{y}{f}}




\end{document}