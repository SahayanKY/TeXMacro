\documentclass{jsarticle}
\usepackage{amsmath}
\usepackage{bm}
\usepackage{listings,jlisting}


\usepackage{derivative}

\renewcommand{\d}{\textrm{d}}

\begin{document}
\derivativesetup{
	d = roman,
}
\begin{table}[p]
\begin{lstlisting}[
	caption=derivativeパッケージのsetup例1,
	label=lstDerivativeSetup1,
	basicstyle=\ttfamily\small
]
\derivativesetup{
	d = roman,
}
\end{lstlisting}
\end{table}
%
%表を挿入
\begin{table}[p]
\centering
\caption{
	常微分マクロの仕様確認。
	コード\ref{lstDerivativeSetup1}による設定を適用した場合の挙動
}
\begin{tabular}{cccc}
No & command & expected & result \\
\hline
1&	\verb|\odr{x}{f}|				& $\displaystyle \frac{\d f}{\d x}$
									& $\displaystyle \odr{x}{f}$ \\[3mm]
2&	\verb|\odr{x}[2]{f}|			& $\displaystyle \frac{\d^2 f}{\d x^2}$
									& $\displaystyle \odr{x}[2]{f}$ \\[3mm]
3&	\verb|\odr{x}[n]{f}|			& $\displaystyle \frac{\d^n f}{\d x^n}$
									& $\displaystyle \odr{x}[n]{f}$ \\[3mm]
4&	\verb|\odr{x_1}{f}|				& $\displaystyle \frac{\d f}{\d x_1}$
									& $\displaystyle \odr{x_1}{f}$ \\[3mm]
5&	\verb|\odr{x_1}{y_2}|			& $\displaystyle \frac{\d y_2}{\d x_1}$
									& $\displaystyle \odr{x_1}{y_2}$ \\[3mm]
6&	\verb|\odr{t}{\bm{u}}|			& $\displaystyle \frac{\d\bm{u}}{\d t}$
									& $\displaystyle \odr{t}{\bm{u}}$ \\[3mm]
7&	\verb|\odr{\bm{x}}{f}|			& $\displaystyle \frac{\d f}{\d\bm{x}}$
									& $\displaystyle \odr{\bm{x}}{f}$ \\[3mm]
\end{tabular}
\end{table}

\derivativesetup{
	d = italic,
}
\begin{table}[p]
\begin{lstlisting}[
	caption=derivativeパッケージのsetup例2,
	label=lstDerivativeSetup2,
	basicstyle=\ttfamily\small
]
\derivativesetup{
	d = italic,
}
\end{lstlisting}
\end{table}
%
%表を挿入
\begin{table}[p]
\centering
\caption{
	常微分マクロの仕様確認。
	コード\ref{lstDerivativeSetup2}による設定を適用した場合の挙動
}
\begin{tabular}{cccc}
No & command & expected & result \\
\hline
1&	\verb|\odr{x}{f}|				& $\displaystyle \frac{d f}{d x}$
									& $\displaystyle \odr{x}{f}$ \\[3mm]
2&	\verb|\odr{x}[2]{f}|			& $\displaystyle \frac{d^2 f}{d x^2}$
									& $\displaystyle \odr{x}[2]{f}$ \\[3mm]
3&	\verb|\odr{x}[n]{f}|			& $\displaystyle \frac{d^n f}{d x^n}$
									& $\displaystyle \odr{x}[n]{f}$ \\[3mm]
4&	\verb|\odr{x_1}{f}|				& $\displaystyle \frac{d f}{d x_1}$
									& $\displaystyle \odr{x_1}{f}$ \\[3mm]
5&	\verb|\odr{x_1}{y_2}|			& $\displaystyle \frac{d y_2}{d x_1}$
									& $\displaystyle \odr{x_1}{y_2}$ \\[3mm]
6&	\verb|\odr{t}{\bm{u}}|			& $\displaystyle \frac{d\bm{u}}{d t}$
									& $\displaystyle \odr{t}{\bm{u}}$ \\[3mm]
7&	\verb|\odr{\bm{x}}{f}|			& $\displaystyle \frac{d f}{d\bm{x}}$
									& $\displaystyle \odr{\bm{x}}{f}$ \\[3mm]
\end{tabular}
\end{table}


%
%表を挿入
\begin{table}[p]
\centering
\caption{
	偏微分マクロの仕様確認
}
\begin{tabular}{cccc}
No & command & expected & result \\
\hline
1&	\verb|\pdr{{x}{f}}|					& $\displaystyle \frac{\partial f}{\partial x}$
										& $\displaystyle \pdr{{x}{f}}$ \\[3mm]
2&	\verb|\pdr{{x}[2]{f}}|				& $\displaystyle \frac{\partial^2 f}{\partial x^2}$
										& $\displaystyle \pdr{{x}[2]{f}}$ \\[3mm]
3&	\verb|\pdr{{x}{y}{f}}|				& $\displaystyle \frac{\partial^2 f}{\partial x \partial y}$
										& $\displaystyle \pdr{{x}{y}{f}}$ \\[3mm]
4&	\verb|\pdr{{x}[2]{y}{f}}|			& $\displaystyle \frac{\partial^3 f}{\partial x^2 \partial y}$
										& $\displaystyle \pdr{{x}[2]{y}{f}}$ \\[3mm]
5&	\verb|\pdr{{x}[2]{y}[3]{f}}|		& $\displaystyle \frac{\partial^5 f}{\partial x^2 \partial y^3}$
										& $\displaystyle \pdr{{x}[2]{y}[3]{f}}$ \\[3mm]
6&	\verb|\pdr{{x_1}{y}}|				& $\displaystyle \frac{\partial y}{\partial x_1}$
										& $\displaystyle \pdr{{x_1}{y}}$ \\[3mm]
7&	\verb|\pdr{{x}{f_2}}|				& $\displaystyle \frac{\partial f_2}{\partial x}$
										& $\displaystyle \pdr{{x}{f_2}}$ \\[3mm]
8&	\verb|\pdr{{x_1}{t}{f}}|			& $\displaystyle \frac{\partial^2 f}{\partial x_1 \partial t}$
										& $\displaystyle \pdr{{x_1}{t}{f}}$ \\[3mm]
9&	\verb|\pdr{{x_1}{x_2}{f}}|			& $\displaystyle \frac{\partial^2 f}{\partial x_1 \partial x_2}$
										& $\displaystyle \pdr{{x_1}{x_2}{f}}$ \\[3mm]
10&	\verb|\pdr{{\bm{x}}{f}}|			& $\displaystyle \frac{\partial f}{\partial \bm{x}}$
										& $\displaystyle \pdr{{\bm{x}}{f}}$ \\[3mm]
11&	\verb|\pdr{{\bm{x}}{\bm{y}}{f}}|	& $\displaystyle \frac{\partial^2 f}{\partial \bm{x} \partial \bm{y}}$
										& $\displaystyle \pdr{{\bm{x}}{\bm{y}}{f}}$ \\[3mm]
12&	\verb|\pdr{{\bm{x}_1}{\bm{y}_2}{f}}|& $\displaystyle \frac{\partial^2 f}{\partial \bm{x}_1 \partial \bm{y}_2}$
										& $\displaystyle \pdr{{\bm{x}_1}{\bm{y}_2}{f}}$ \\[3mm]
13&	\verb|\pdr{{x}[2]{\bm{u}}}|			& $\displaystyle \frac{\partial^2 \bm{u}}{\partial x^2}$
										& $\displaystyle \pdr{{x}[2]{\bm{u}}}$ \\[3mm]
14&	\verb|\pdr{{x}{\bm{u}}}+\pdr{{y}{\bm{u}}}| & $\displaystyle \frac{\partial \bm{u}}{\partial x} + \frac{\partial \bm{u}}{\partial y}$
										& $\displaystyle \pdr{{x}{\bm{u}}}+\pdr{{y}{\bm{u}}}$ \\[3mm]
15&	\verb|\pdr{{x}{}}| 					& $\displaystyle \frac{\partial}{\partial x}$
										& $\displaystyle \pdr{{x}{}}$ \\[3mm]
16&	\verb|\pdr{{x_1}{}}|				& $\displaystyle \frac{\partial}{\partial x_1}$
										& $\displaystyle \pdr{{x_1}{}}$ \\[3mm]
17&	\verb|\pdr{{\bm{x}}{}}|				& $\displaystyle \frac{\partial}{\partial \bm{x}}$
										& $\displaystyle \pdr{{\bm{x}}{}}$ \\[3mm]
\end{tabular}
\end{table}
%
%表を挿入
\begin{table}[p]
\centering
\caption{
	偏微分マクロの仕様確認。
	階数の自動計算抑制
}
\begin{tabular}{cccc}
No & command & expected & result \\
\hline
1& \verb|\pdr*{{x}[n]{f}[n]}|			& $\displaystyle \frac{\partial^n f}{\partial x^n}$
										& $\displaystyle \pdr*{{x}[n]{f}[n]}$ \\[3mm]
2& \verb|\pdr*{{x}[n]{}[n]}|			& $\displaystyle \frac{\partial^n}{\partial x^n}$
										& $\displaystyle \pdr*{{x}[n]{f}[n]}$ \\[3mm]
3& \verb|\pdr*{{x}[n]{y}[m]{f}[n+m]}|	& $\displaystyle \frac{\partial^{n+m} f}{\partial x^n \partial y^m}$
										& $\displaystyle \pdr*{{x}[n]{y}[m]{f}[n+m]}$ \\[3mm]
4& \verb|\pdr*{{x}[1]{y}[1]{f}[1+1]}|	& $\displaystyle \frac{\partial^{1+1} f}{\partial x^1 \partial y^1}$
										& $\displaystyle \pdr*{{x}[1]{y}[1]{f}[1+1]}$ \\[3mm]
5& \verb|\pdr*{{x}{y}[n]{f}[n+1]}|		& $\displaystyle \frac{\partial^{n+1} f}{\partial x \partial y^n}$
										& $\displaystyle \pdr*{{x}{y}[n]{f}[n+1]}$ \\[3mm]
\end{tabular}
\end{table}
%
%表を挿入
\begin{table}[p]
\centering
\caption{
	偏微分マクロ(リーマン幾何学の記法)の仕様確認
}
\begin{tabular}{cccc}
No & command & expected & result \\
\hline
1& \verb|\pdrr{{x}{f}}|					& $\displaystyle \partial_{x} f$
										& $\displaystyle \pdrr{{x}{f}}$ \\[3mm]
2& \verb|\pdrr{{i}{j}{f}}|				& $\displaystyle \partial_{i}\partial_{j} f$
										& $\displaystyle \pdrr{{i}{j}{f}}$ \\[3mm]
3& \verb|\pdrr{{i}*{j}{f}}|				& $\displaystyle \partial_{i}\partial^{j} f$
										& $\displaystyle \pdrr{{i}*{j}{f}}$ \\[3mm]
4& \verb|\pdrr{*{i}{j}{f}}|				& $\displaystyle \partial^{i}\partial_{j} f$
										& $\displaystyle \pdrr{*{i}{j}{f}}$ \\[3mm]
\end{tabular}
\end{table}

\derivativesetup{
	nabla-deco-1 = ',
	nabla-deco-2 = '',
	nabla-deco-3 = _{\bm{r}'''},
	laplace = delta,
}
\begin{table}[p]
\begin{lstlisting}[
	caption=derivativeパッケージのsetup例3,
	label=lstDerivativeSetup3,
	basicstyle=\ttfamily\small
]
\derivativesetup{
	nabla-deco-1 = ',
	nabla-deco-2 = '',
	nabla-deco-3 = _{\bm{r}'''},
	laplace = delta,
}
\end{lstlisting}
\end{table}
%
%表を挿入
\begin{table}[p]
\centering
\caption{
	ベクトル演算子マクロの仕様確認。
	コード\ref{lstDerivativeSetup3}による設定を適用した場合の挙動
}
\begin{tabular}{cccc}
No & command & expected & result \\
\hline
1& \verb|\grad{f}|						& $\displaystyle \nabla f$
										& $\displaystyle \grad{f}$ \\[3mm]
2& \verb|\div\bm{v}|					& $\displaystyle \nabla\cdot\bm{v}$
										& $\displaystyle \div\bm{v}$ \\[3mm]
3& \verb|\rot\bm{v}|					& $\displaystyle \nabla\times\bm{v}$
										& $\displaystyle \rot\bm{v}$ \\[3mm]
4& \verb|\curl\bm{v}|					& $\displaystyle \nabla\times\bm{v}$
										& $\displaystyle \curl\bm{v}$ \\[3mm]
5& \verb|\laplace{f}|					& $\displaystyle \Delta f$
										& $\displaystyle \laplace{f}$ \\[3mm]
6& \verb|\laplaced{f}|					& $\displaystyle \Delta f$
										& $\displaystyle \laplaced{f}$ \\[3mm]
7& \verb|\laplacen{f}|					& $\displaystyle \nabla^2 f$
										& $\displaystyle \laplacen{f}$ \\[3mm]
8& \verb|\hesse{f}|						& $\displaystyle \nabla\otimes\nabla f$
										& $\displaystyle \hesse{f}$ \\[3mm]
9& \verb|\gradr{f}|						& $\displaystyle \textrm{grad}\, f$
										& $\displaystyle \gradr{f}$ \\[3mm]
10& \verb|\divr{\bm{v}}|				& $\displaystyle \textrm{div}\, \bm{v}$
										& $\displaystyle \divr{\bm{v}}$ \\[3mm]
11& \verb|\rotr{\bm{v}}|				& $\displaystyle \textrm{rot}\, \bm{v}$
										& $\displaystyle \rotr{\bm{v}}$ \\[3mm]
12& \verb|\curlr{\bm{v}}|				& $\displaystyle \textrm{curl}\, \bm{v}$
										& $\displaystyle \curlr{\bm{v}}$ \\[3mm]
13& \verb|\grad'{f}|					& $\displaystyle \nabla' f$
										& $\displaystyle \grad'{f}$ \\[3mm]
14& \verb|\grad''{f}|					& $\displaystyle \nabla'' f$
										& $\displaystyle \grad''{f}$ \\[3mm]
15& \verb|\grad'''(\frac{f}{g})|		& $\displaystyle \nabla_{\bm{r}'''} \left(\frac{f}{g}\right)$
										& $\displaystyle \grad'''(\frac{f}{g})$ \\[3mm]
16& \verb|\gradr''[f]|					& $\displaystyle \textrm{grad}'' \left[f\right]$
										& $\displaystyle \gradr''[f]$ \\[3mm]
17& \verb|\hesse''!{f}|					& $\displaystyle \nabla''\otimes\nabla'' \left\{f\right\}$
										& $\displaystyle \hesse''!{f}$ \\[3mm]
\end{tabular}
\end{table}





\end{document}